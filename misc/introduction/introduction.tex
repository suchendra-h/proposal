\chapter{Introduction}

\section{Benefits of Architecture Recovery}
\begin{itemize}
    \item New members of a development team can benefit from the architecture by having access to the group memory of all the artifacts generated during the development process. Tools such as Hipikat \cite{Cubranic2005} can generate such a memory by analyzing which developers have the most contribution for a component. Leveraging this information, new developers will know which team member to contact in case of any questions.
    \item Project managers can benefit from the information on which parts of the project is actively changing and how this change is affecting the structure of the project \cite{leh80}.
    \item By creating a mapping between the changes and author identifiers, project managers can also understand who has worked on an artifact and thus, is more knowledgeable about a problem at hand for the same artifact \cite{Girba05}.
    \item Identify the distance between the requirement and implementation of the system, i.e., as-is-architecture and as-it-should-be architecture \cite{eixelberger98}
    \item Architecture Recovery can help learn the structure of a program and how it satisfies the domain needs 
\end{itemize}

% Why not? \cite{svetinovic18}
% \begin{itemize}
%     \item Most developers like to concentrate exclusively on programming.
%     \item Design documents are almost never in synchronization with code.
%     \item During active development, lot of time is spent updating design documents.
%     \item The produced documents are not used much by programmers.
%     \item Tool support for roundtrip engineering is not good enough.
% \end{itemize}

% \section{To read list}
% \begin{itemize}
%     \item Architecture  description  language  (ADL) or Architecture Structure Description Language (ASDL).
%     \item Configuration management (CM) systems [Burrows and Wesley 2001]
%     \item 
% \end{itemize}
